\documentclass[lettersize,journal,twoside]{IEEEtran} 
\usepackage[inline]{enumitem}
\usepackage{amsmath,amsfonts}
\usepackage{amssymb}
\usepackage{algorithmic}
\usepackage{algorithm}
\usepackage{array}
%\usepackage[caption=false,font=normalsize,labelfont=sf,textfont=sf]{subfig}
\usepackage{textcomp}
\usepackage{stfloats}
\usepackage{url}
\usepackage{verbatim}
\usepackage{graphicx}
\usepackage{tikz}
\usepackage[font=footnotesize]{caption}
\usepackage[font=footnotesize]{subcaption}
\usepackage[noadjust]{cite}

%% package for urls
\usepackage{url}

%% hyperref
% and an override to make hyperref work with ieeeconf.cls
\makeatletter
\let\NAT@parse\undefined
\makeatother
\usepackage[pagebackref=true,breaklinks=true,colorlinks,bookmarks=false]{hyperref}
\makeatletter
\newcommand*{\textlabel}[2]{%
  \edef\@currentlabel{#1}% Set target label
  \phantomsection% Correct hyper reference link
  #1\label{#2}% Print and store label
}
\makeatother

\usepackage{textpos}
\usepackage{amsthm}
\usepackage{xcolor}
\colorlet{RED}{red}

\usepackage{tikz}
\usepackage[scaled]{helvet}

\AddToHook{shipout/foreground}{
  \begin{tikzpicture}[remember picture,overlay]
    \node[red,rotate=45,scale=10,opacity=0.2] at (current page.center) {\small\fontfamily{phv}\selectfont};
    %IN~PREPARATION};
    %UNDER~REVIEW};   
  \end{tikzpicture}
}

%% correct bad hyphenation here
\hyphenation{obs-tacles sur-roundings}

\renewcommand{\qedsymbol}{$\blacksquare$}

\theoremstyle{definition}
\newtheorem{thm}{Theorem}[section]
\newtheorem{lem}[thm]{Lemma}
\newtheorem{prop}[thm]{Proposition}
\newtheorem{assm}[thm]{Assumption}
\newtheorem{cor}{Corollary}
\newtheorem{conj}{Conjecture}[section]
\newtheorem{defn}{Definition}[section]
\newtheorem{exmp}{Example}[section]
\newtheorem{pb}{Problem}[section]
\newtheorem{rem}{Remark}
\newtheorem{obs}{Observation}
\newtheorem*{ctb}{Contribution}

\DeclareMathOperator*{\argmax}{arg\,max}
\DeclareMathOperator*{\argmin}{arg\,min}

\makeatletter
\newcommand\notsotiny{\@setfontsize\notsotiny\@vipt\@viipt}
\makeatother

\renewcommand\citepunct{,\hspace*{.8ex}}
\renewcommand*{\citedash}{--}

\begin{document}
\bstctlcite{IEEEexample:BSTcontrol}

\title{Energy-Aware Ergodic Search: Continuous Long-Term Exploration for Multiagent Systems}

\author{Adam Seewald${}^\text{1}$, Marvin Chanc{\'a}n${}^\text{1}$, Cameron J. Lerch${}^\text{1}$, Hector Castillo${}^\text{1}$, Aaron M. Dollar${}^\text{1}$, and Ian Abraham${}^\text{1}$%~\IEEEmembership{Staff,~IEEE,}
        % <-this % stops a space
  %\thanks{Manuscript received: Month, Day, Year; Revised Month, Day, Year; Accepted Month, Day, Year.}%Use only for final RAL version
  %\thanks{This paper was recommended for publication by Editor Editor A. Name upon evaluation of the Associate Editor and Reviewers' comments.} %Use only for final RAL version
  \thanks{${}^\text{1}$A.\hspace*{.4ex}S., M.\hspace*{.4ex}C., C.\hspace*{.4ex}J.\hspace*{.4ex}L., H.\hspace*{.4ex}C., A.\hspace*{.4ex}M.\hspace*{.4ex}D., and I.\hspace*{.4ex}A. are with the Department of Mechanical Engineering and Materials Science, Yale University, CT, USA. Email: {\tt\footnotesize \href{mailto:adam.seewald@yale.edu}{adam.seewald@yale.edu};}}
  %\thanks{${}^\text{3}$A.\hspace*{.4ex}M.\hspace*{.4ex}W. is with the Department of Mechanical Engineering, Massachusetts Institute of Technology, MA, USA;}
  %\thanks{${}^\text{4}$V.\hspace*{.4ex}E. and B.\hspace*{.4ex}B. are currently unaffiliated. ${}^\text{2, 3, 4}$C.\hspace*{.4ex}M.\hspace*{.4ex}C., A.\hspace*{.4ex}M.\hspace*{.4ex}W., V.\hspace*{.4ex}E., and B.\hspace*{.4ex}B. performed the work while affiliated with Yale University.}
  %\thanks{Digital Object Identifier (DOI): see top of this page.}
}

% The paper headers
\markboth{%Journal of \LaTeX\ Class Files,~Vol.~14, No.~8, August~2021
%Manuscript~submitted~for~publication. Version~\textnumero~1, 
Manuscript in preparation. %\underline{Not for distribution}. 
July~2023
}%
{Seewald \MakeLowercase{\textit{et al.}}: Ergodic Search for Long-Term Exploration}

\IEEEpubid{%0000--0000/00\$00.00~\copyright~2023 IEEE
}
% Remember, if you use this you must call \IEEEpubidadjcol in the second
% column for its text to clear the IEEEpubid mark.

\maketitle

%\vspace*{-1cm}
\begin{abstract} 
---
\end{abstract}


%%%%%%%%%%%%%%%%%%%%
\begin{IEEEkeywords}
  Motion and Path Planning; Energy and Environment-Aware Autonomation.
\end{IEEEkeywords}


%%%%%%%%%%%%%%%%%%%%%%
\section{Introduction}

\IEEEPARstart{A}{a} 
\IEEEpubidadjcol

---


%%%%%%%%%%%%%%%%%%%%%%%%%%%%%
\section{Problem Formulation}
\noindent
The problem addressed in this work is that of exploring a bounded space with multiple agents, continuosly, and proportionally to a spatial distribution. To achieve the latter, canonical ergodic search~\cite{mathew2011metrics} derives the agent's control so that its trajectory maximizes an ergodic metric defined in the spectral domain~\cite{calinon2020mixture}.

\begin{pb}[Ergodic search]\label{pb:ergo}
  Consider a bounded space $\mathcal{Q}\subset\mathbb{R}^D$ of dimension $D$ with $D\in\mathbb{N}_{>0}$ and a spatial distribution $\phi$. \textit{Ergodic search problem} is the problem of deriving a control action $\mathbf{u}(t)\in\mathcal{U}\subset\mathbb{R}^V$ with $V\in\mathbb{N}_{>0}$ so that the trajectory $\mathbf{q}(t)\in\mathcal{Q}$ is proportional to the distribution $\phi$.
\end{pb}

Here the notation $\mathbb{R}$ and $\mathbb{N}$ indicates reals and naturals, $\mathbb{N}_{>0}$ strictly naturals. Bold notation is used for vectors.

We extend the canonical erogidc search problem, to multi-agent continuous ergodic search, i.e., exploration with multiple robots under spatial distribution and battery constraints.

\begin{pb}[Multi-agent continuous ergodic search]\label{pb:enerergo}
  Consider a set of $n$ agents $\boldsymbol{\alpha}:=\{\alpha_1,\alpha_2,\dots,\alpha_n\}$, a bounded space $\mathcal{Q}$, and a spatial distribution $\phi$ similarly to Problem~\ref{pb:ergo}. We are interested in deriving each agent ${}^j\alpha$ control actions ${}^j\mathbf{u}(t)$ so that its trajectory ${}^j\mathbf{q}(t)$ is proportional to the distribution $\phi$.
\end{pb}


%%%%%%%%%%%%%%%%%%
\section{Methods}
\noindent

---

\subsection{Ergodic search}\label{sec:ergosearch}
\noindent
For the purposes of defining the spatial distribution, let us consider a Gaussian mixture model (GMM)
\begin{equation}\label{eq:gmm}
  \phi(\boldsymbol{\delta},\mathbf{q}):=\sum_{k=1}^{m}\delta_k\,\mathcal{N}(\mathbf{q}\,|\,\mu_k,\Sigma_k),
\end{equation} 
composed of $m$ Gaussians. Each has a coveriance matrix ${\Sigma_k}\in\mathbb{R}^{D\times D}$, a center $\mu_k\in\mathcal{Q}$, and a positive mixing coefficient $\delta_k\in\boldsymbol{\delta}$ such that the sum of the $\delta$s is less or equal to one. They indicate how well is each Guassian in the GMM considered. 

The goal of ergodic search is to minimize an ergodic metric
\begin{equation}\label{eq:ergmetric}
  \mathcal{E}(\boldsymbol{\delta},\mathbf{q}(t),t):=\frac{1}{2}\sum_{k\in\mathcal{K}}\Lambda_k \big( c_k(\mathbf{q}(t),t)-\phi_k(\boldsymbol{\delta},t) \big)^2,
\end{equation}
where $\phi_k$ are coefficients derived utilizing the Fourier series on the spatial distribution $\phi$ and $c_k$ on the trajectory $\mathbf{q}(t)$. They are detailed in Equation~(\ref{eq:phik}) and (\ref{eq:ck}) respectively.
$\Lambda_k$ is a weight factor, i.e., if 
\begin{equation}
  \Lambda_k=(1+\lVert k\rVert^2)^{(-D-1)/2},
\end{equation}
lower frequencies have more weight~\cite{miller2016ergodic}.
$\mathcal{K}\in\mathbb{N}^D$ is a set index vectors that covers $[K]\times\cdots\times[K]\in\mathbb{N}^{K^D}$ 
where $K$ is a given number of frequencies including the fundamental frequency. The notation $[K]$ indicates positive naturals up to $K$.

The coefficients $c_k$ are derived using the Fourier series basis function. If we consider the trigonometric form
\begin{equation}\label{eq:ck}
  \begin{split}
    c_k(\mathbf{q}(t),t):=\int_{\mathcal{T}}\frac{1}{L^D}\prod_{d\in[D]_{>0}\hspace*{-1ex}}(& \cos(k_d\,\mathbf{q}_d(\tau)\,\psi)\\[-2ex]
    &\vspace*{-4ex}-i\sin(k_d\,\mathbf{q}_d(\tau)\,\psi) )\,d\tau/t,
  \end{split}
\end{equation}
$c_k$ is then evaluated per each $k$ in $\mathcal{K}$ in Eq.~(\ref{eq:ergmetric}).
$\psi$ is $2\pi/L$ for a given period $L\in\mathbb{R}_{>0}$, $i$ is the imaginary unit, $k_d$ is the $d$th item of $k$, and $\mathbf{q}_d$ the $d$th item of $\mathbf{q}$.

$\mathcal{T}$ is built so that the integration is between $\tau=t_0$ and $t$, and the notation $[D]_{>0}$ indicates strictly positive naturals up to $D$.

For the purposes of deriving the coefficients $\phi_k$, let us consider the GMM model in Eq.~(\ref{eq:gmm}) on a search space $\mathcal{Q}$. The space is further bounded to a symmetric set $[-L/2,L/2]^D$ since the Gaussians are symmetric about the zero axes. The resulting new model is then
\begin{equation}
  \Phi(\boldsymbol{\delta},\mathbf{q}):=\sum_{d\in[2^D]_{>0}}\sum_{k=1}^{m}\delta_k\,\mathcal{N}(\mathbf{q}\,|\,A_d\mu_k,A_d\Sigma_k A_d^T)/2^D,
\end{equation}
where $A_d\in\mathbb{R}^{D\times D}$ are linear transformation matrices~\cite{calinon2020mixture}. Let us call the integrand in Eq.~(\ref{eq:ck}) $c:\mathcal{Q}\longrightarrow \mathbb{R}^K$. It maps the space to spectral domain. The equivalent of Eq.~(\ref{eq:ck}) for the spatial distribution can be then expressed
\begin{equation}\label{eq:phik}
  \phi_k(\boldsymbol{\delta},t):=\int_{\mathcal{Q}} \Phi(\boldsymbol{\delta},\mathbf{q})\,c(\mathbf{q})\,\,d\mathbf{q}.
\end{equation}

$\mathcal{Q}$ is build so that the integration is withing the points of the bounded symmetric set $\mathbf{q}\in[-L/2,L/2]^D$.

Let us first formulate the solution for Problem~\ref{pb:ergo}, borrowed by canonical ergodic search~\cite{ayvali2017ergodic}.
If the robot's dynamics is described by a generic differential equation $\dot{\mathbf{q}}(t)=f\big({\small\mathbf{q}(t),}$ ${\small\mathbf{u}(t),t})$, an optimal control problem (OCP) that selects an ergodic control action can be formulated
\begin{subequations}\label{eq:ocpergo}\begin{align}
  \min_{\mathbf{q}(t),\mathbf{u}(t)}&{\mathcal{E}(\boldsymbol{\delta},\mathbf{q}(t),t_f)}+\int_{\mathcal{T}}\mathbf{u}(\tau)^TR\mathbf{u}(\tau)\,d\tau,\label{eq:ocpergomin}\\
  \text{s.t. }\dot{\mathbf{q}}&=f(\mathbf{q}(t),\mathbf{u}(t),t),\\
  \mathbf{q}&(t)\in\mathcal{Q},\,\mathbf{u}(t)\in\mathcal{U},\\
  \mathbf{q}&(t_0), \mathbf{q}(t_f)\text{ given},
\end{align}\end{subequations}
where the ergodic metric is derived in Eq.~(\ref{eq:ergmetric}), $R\in\mathbb{R}^{V\times V}$ is a control penalizing diagonal positive-definite matrix, and $t_0, t_f$ are respectively the first and last time instants. 
$\mathcal{T}$ is $[t_0, t_f)$.


To formulate the solution to Pb.~\ref{pb:enerergo}, let us first extend the OCP in Eq.~(\ref{eq:ocpergo}) to multiagent systems. Eq.~(\ref{eq:ocpergomin}) becomes
\begin{equation}
  \min_{\square}\,\,\,{\sum_{k=1}^{n}\left(\mathcal{E}(\boldsymbol{\delta},{}^k\hspace*{-.2ex}\mathbf{q}(t),t)+\int_{\mathcal{T}_k}{}^k\hspace*{-.1ex}\mathbf{u}(\tau)^TR_k\,{}^k\hspace*{-.1ex}\mathbf{u}(\tau)\,d\tau\right)},
\end{equation}
where the ergodic metrics and the control penalizing term $R_k$ are now agent-specifc. The term $\square$ is ${}^1\mathbf{q}(t),{}^2\mathbf{q}(t),\dots,$ ${}^n\mathbf{q}(t),{}^1\mathbf{u}(t),{}^2\mathbf{u}(t),\dots{}^n\mathbf{u}(t)$. $\mathcal{T}_k$ is $[{}^kt_0, {}^kt_f)$, i.e., different agents might have different duration.

\subsection{Battery modeling}
\noindent

---

%%%%%%%%%%%%%%%%%%%%%%%%%%%%%%
\section{Experimental Results}
\noindent

---

\begin{figure*}[p]
  \begin{minipage}[t]{1\columnwidth}

    \input{figures/trajs.pdf_tex}
  \end{minipage}
  \hspace{.42cm}
  \begin{minipage}[t]{.93\columnwidth}
    \vspace*{-4.2cm}
    \caption{.   .   .   .   .   .   .   .   .   .   .   .   .   .   .   .   .   .   .   .   .   .   .   .   .   .   .   .   .   .   .   .   .   .   .   .   .   .   .   .   .   .   .   .   .   .   .   .   .   .   .   .   .   .   .   .   .   .   .   .   .   .   .   .   .   .   .   .   .   .   .   .   .   .   .   .   .   .   .   .   .   .   .   .   .   .   .   .   .   .   .   .   .   .   .   .   .   .   .   .   .   .   .   .   .   .   .   .   .   .   .   .   .   .   .   .   .   .   .   .   .   .   .   .   .   .   .   .   .   .   .   .   .   .   .   .   .   .   .   .   .   .   .   .   .   .   .   .   .   .   .   .   .   .   .   .   .   .   .   .   .   .   .   .   .   .   .   .   .   .   .   .   .   .   .   .   .   .   .   .   .   .   .   .   .   .   .   .   .   .   .   .   .   .   .   .   .   .   .   .   .   .   .   .   .   .   .   .   .   .   .   .   .   .   .   .   .   .   .   .   .   .   .   .   .   .   .   .   .   .   .   .   .   .   .   .   .   .   .   .   .   .   .   .   .   .   .   .   .   .   .   .   .   .   .   .   .   .   .   .   .   .   .   .   .   .   .   .   .   .   .   .   .   .   .   .   .   .   .   .   .   .   .   .   .   .   .   .   .   .   .   .   .   .   .   .   .   .   .   .   .   .   .   .   .   .   .   .   .   .   .   .   .   .   .   .   .   .   .   .   .   .   .   .   .   .   .   .   .   .   .   .   .   .   .   .   .   .   .   .   .   .   .   .   .   .   .   .   .   .   .   .   .   .   .   .   .   .   .   .   .   .   .   .   .   .   .   .   .   .   .   .   .   .   .   .   .   .   .   .   .   .   .   .   .   .   .   .   .   .   .   .   .   .   .   .   .   .   .   .   .   .   .   .   .   .   .   .   .   .   .   .   .   .   .   .   .   .   .   .   .   .   .   .   .   .   .   .   .   .   .   .   .   .   .   .   .   .   .   .   .   .   .   .   .   .   .   .   .   .   .   .   .   .   .   .   .   .   .   .   .   .   .   .   .   .   .   .   .   .   .   .   .   .   .   .   .   .   .   .   .   .   .   .   .   .   .   .   .   .   .   .   .   .   .   .   .   .   .   .   .   .   .   .   .   .   .   .   .   .   .   .   .   .   .   .   .   .   .   .   .   .   .   .   .   .   .   .   .   .   .   .   .   .   .   .   .   .   .   .   .   .   .   .   .   .   .   .   .   .   .   .   .   .   .   .   .   .   .   .   .   .   .   .   .   .   .   .   .   .   .   .   .   .   .   .   .   .   .   .   .   .   .   .   .   .   .   .   .   .   .   .   .   .   .   .   .   .   .   .   .   .   .   .   .   .   .   .   .   .   .   .   .   .   .   .   .   .   .   .   .   .   .   .   .   .   .   .   .   .   .   .   .   .   .   .   .   .   .   .   .   .   .   .   .   .   .   .   .   .   .   .   .   .   .   .}
  \end{minipage}
\end{figure*}

%%%%%%%%%%%%%%%%%%%%%%%%%%%%%%%%%%%%%%%%%%
\section{Conclusion and Future Directions}
\noindent

---


{\small
\bibliographystyle{IEEEtran} 
\bibliography{enerergo}
}

\end{document}


