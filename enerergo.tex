\documentclass[lettersize,journal,twoside]{IEEEtran} 
\usepackage[inline]{enumitem}
\usepackage{amsmath,amsfonts}
\usepackage{amssymb}
\usepackage{algorithmic}
\usepackage{algorithm}
\usepackage{array}
%\usepackage[caption=false,font=normalsize,labelfont=sf,textfont=sf]{subfig}
\usepackage{textcomp}
\usepackage{stfloats}
\usepackage{url}
\usepackage{verbatim}
\usepackage{graphicx}
\usepackage{tikz}
\usepackage[font=footnotesize]{caption}
\usepackage[font=footnotesize]{subcaption}
\usepackage[noadjust]{cite}

%% package for urls
\usepackage{url}

%% hyperref
% and an override to make hyperref work with ieeeconf.cls
\makeatletter
\let\NAT@parse\undefined
\makeatother
\usepackage[pagebackref=true,breaklinks=true,colorlinks,bookmarks=false]{hyperref}
\makeatletter
\newcommand*{\textlabel}[2]{%
  \edef\@currentlabel{#1}% Set target label
  \phantomsection% Correct hyper reference link
  #1\label{#2}% Print and store label
}
\makeatother

\usepackage{textpos}
\usepackage{amsthm}
\usepackage{xcolor}
\colorlet{RED}{red}

\usepackage{tikz}
\usepackage[scaled]{helvet}

\AddToHook{shipout/foreground}{
  \begin{tikzpicture}[remember picture,overlay]
    \node[red,rotate=45,scale=10,opacity=0.2] at (current page.center) {\small\fontfamily{phv}\selectfont};
    %IN~PREPARATION};
    %UNDER~REVIEW};   
  \end{tikzpicture}
}

%% correct bad hyphenation here
\hyphenation{obs-tacles sur-roundings}

\renewcommand{\qedsymbol}{$\blacksquare$}

\theoremstyle{definition}
\newtheorem{thm}{Theorem}[section]
\newtheorem{lem}[thm]{Lemma}
\newtheorem{prop}[thm]{Proposition}
\newtheorem{assm}[thm]{Assumption}
\newtheorem{cor}{Corollary}
\newtheorem{conj}{Conjecture}[section]
\newtheorem{defn}{Definition}[section]
\newtheorem{exmp}{Example}[section]
\newtheorem*{pb}{Problem}%[section]
\newtheorem{rem}{Remark}
\newtheorem{obs}{Observation}
\newtheorem*{ctb}{Contribution}

\DeclareMathOperator*{\argmax}{arg\,max}
\DeclareMathOperator*{\argmin}{arg\,min}

\makeatletter
\newcommand\notsotiny{\@setfontsize\notsotiny\@vipt\@viipt}
\makeatother

\renewcommand\citepunct{,\hspace*{.8ex}}
\renewcommand*{\citedash}{--}

\begin{document}
\bstctlcite{IEEEexample:BSTcontrol}

\title{Energy-Aware Ergodic Search: Continuous Long-Term Exploration for Multiagent Systems}

\author{Adam Seewald${}^\text{1}$, Marvin Chanc{\'a}n${}^\text{1}$, Cameron J. Lerch${}^\text{1}$, Hector Castillo${}^\text{1}$, Aaron M. Dollar${}^\text{1}$, and Ian Abraham${}^\text{1}$%~\IEEEmembership{Staff,~IEEE,}
        % <-this % stops a space
  %\thanks{Manuscript received: Month, Day, Year; Revised Month, Day, Year; Accepted Month, Day, Year.}%Use only for final RAL version
  %\thanks{This paper was recommended for publication by Editor Editor A. Name upon evaluation of the Associate Editor and Reviewers' comments.} %Use only for final RAL version
  \thanks{${}^\text{1}$A.\hspace*{.4ex}S., M.\hspace*{.4ex}C., C.\hspace*{.4ex}J.\hspace*{.4ex}L., H.\hspace*{.4ex}C., A.\hspace*{.4ex}M.\hspace*{.4ex}D., and I.\hspace*{.4ex}A. are with the Department of Mechanical Engineering and Materials Science, Yale University, CT, USA. Email: {\tt\footnotesize \href{mailto:adam.seewald@yale.edu}{adam.seewald@yale.edu};}}
  %\thanks{${}^\text{3}$A.\hspace*{.4ex}M.\hspace*{.4ex}W. is with the Department of Mechanical Engineering, Massachusetts Institute of Technology, MA, USA;}
  %\thanks{${}^\text{4}$V.\hspace*{.4ex}E. and B.\hspace*{.4ex}B. are currently unaffiliated. ${}^\text{2, 3, 4}$C.\hspace*{.4ex}M.\hspace*{.4ex}C., A.\hspace*{.4ex}M.\hspace*{.4ex}W., V.\hspace*{.4ex}E., and B.\hspace*{.4ex}B. performed the work while affiliated with Yale University.}
  %\thanks{Digital Object Identifier (DOI): see top of this page.}
}

% The paper headers
\markboth{%Journal of \LaTeX\ Class Files,~Vol.~14, No.~8, August~2021
%Manuscript~submitted~for~publication. Version~\textnumero~1, 
Manuscript in preparation. %\underline{Not for distribution}. 
July~2023
}%
{Seewald \MakeLowercase{\textit{et al.}}: Ergodic Search for Long-Term Exploration}

\IEEEpubid{%0000--0000/00\$00.00~\copyright~2023 IEEE
}
% Remember, if you use this you must call \IEEEpubidadjcol in the second
% column for its text to clear the IEEEpubid mark.

\maketitle

%\vspace*{-1cm}
\begin{abstract} 
---
\end{abstract}


%%%%%%%%%%%%%%%%%%%%
\begin{IEEEkeywords}
  Motion and Path Planning; Energy and Environment-Aware Autonomation.
\end{IEEEkeywords}


%%%%%%%%%%%%%%%%%%%%%%
\section{Introduction}

\IEEEPARstart{A}{a} 
\IEEEpubidadjcol

---


%%%%%%%%%%%%%%%%%%%%%%%%%%%%%
\section{Problem Formulation}
\noindent
The problem addressed in this work is that of exploring a bounded space with multiple agents, continuosly, and proportionally to a spatial distribution. To achieve the latter, canonical ergodic search~\cite{mathew2011metrics} derives the agent's control so that its trajectory maximizes an ergodic metric defined in the spectral domain~\cite{calinon2020mixture}.

\begin{pb}[Ergodic search]\label{pb:ergo}
  Consider a bounded space $\mathcal{Q}\subset\mathbb{R}^D$ of dimension $D\in\mathbb{N}_{>0}$ and a spatial distribution $\phi$. \textit{Ergodic search problem} is the problem of deriving a control action $\mathbf{u}(t)\in\mathcal{U}$ so that the trajectory $\mathbf{q}(t)\in\mathcal{Q}$ is proportional to the distribution $\phi$.
\end{pb}

Here the notation $\mathbb{R}$ and $\mathbb{N}$ indicates reals and naturals, $\mathbb{N}_{>0}$ strictly naturals. Bold notation is used for vectors.

We extend the canonical erogidc search problem, to multi-agent continuous ergodic search, i.e., exploration with multiple robots under spatial distribution and battery constraints.

\begin{pb}[Multi-agent continuous ergodic search]
  Consider a set of $n$ agents $\boldsymbol{\alpha}:=\{\alpha_1,\alpha_2,\dots,\alpha_n\}$, a bounded space $\mathcal{Q}$, and a spatial distribution $\phi$ similarly to Problem~\ref{pb:ergo}. We are interested in deriving each agent ${}^j\alpha$ control actions ${}^j\mathbf{u}(t)$ so that its trajectory ${}^j\mathbf{q}(t)$ is proportional to the distribution $\phi$.
\end{pb}


%%%%%%%%%%%%%%%%%%
\section{Methods}
\noindent

---

\subsection{Ergodic search}\label{sec:ergosearch}
\noindent
For the purposes of defining the spatial distribution, let us consider a Gaussian mixture model (GMM)
\begin{equation}\label{eq:gmm}
  \phi(\boldsymbol{\delta},\mathbf{q}(t)):=\sum_{k=1}^{n}\delta_k\,\mathcal{N}(\mathbf{q}(t)\,|\,\mu_k,\Sigma_k),
\end{equation} 
composed of $n$ Gaussians. Each has a coveriance matrix ${\Sigma_k}\in\mathbb{R}^{D\times D}$, a center $\mu_k\in\mathcal{Q}$, and a positive mixing coefficient $\delta_k\in\boldsymbol{\delta}$ such that the sum of the $\delta$s is one.

The goal of ergodic search is to minimize an ergodic metric
\begin{equation}\label{eq:ergmetric}
  \mathcal{E}(\mathbf{q}(t),\phi):=\frac{1}{2}\sum_{k\in\mathcal{K}}\Lambda_k(c_k-\phi_k)^2,
\end{equation}
where $\phi_k$ are coefficients derived utilizing the Fourier series on the spatial distribution $\phi$ and $c_k$ on the trajectory $\mathbf{q}$.
$\mathcal{K}$ is a set of vectors in $\mathbb{N}^D$  built so that it contains the indices of all the frequencies, i.e., if there are $k$ frequencies in 2D including the fundamental frequency, $\mathcal{K}$ is 
$\small\{\,\begin{bmatrix}0 &\hspace*{-.7ex} \cdots \hspace*{-.7ex}& k\end{bmatrix}^T,$
$\small\begin{bmatrix}0 &\hspace*{-.7ex} \cdots \hspace*{-.7ex}& 0\end{bmatrix}^T\},\dots,\hspace*{-.5ex}\{\,\begin{bmatrix}0 &\hspace*{-.7ex} \cdots \hspace*{-.7ex}& k\end{bmatrix}^T,\begin{bmatrix}k &\hspace*{-.7ex} \cdots \hspace*{-.7ex}& k\end{bmatrix}^T\}$.
Finally, $\Lambda_k$ is a weight factor, i.e., if $\Lambda_k$ is $(1+\lVert k\rVert^2)^{(-D-1)/2}$ lower fre- quencies have more weight.

The coefficients $c_k$ are derived using the Fourier series basis function. If we consider the trigonometric form
\begin{equation}
  \begin{split}
    c_k(\mathbf{q},t):=\int_{\mathcal{T}}\frac{1}{L^D}\prod_{d\in[D]}(& \cos(k_d\,\mathbf{q}_d(\tau)\,\psi)\\[-2ex]
    &\vspace*{-4ex}-i\sin(k_d\,\mathbf{q}_d(\tau)\,\psi) )\,d\tau/t,
  \end{split}
\end{equation}
$c_k$ is then evaluated per each $k$ in $\mathcal{K}$ in Eq.~(\ref{eq:ergmetric}).
$\psi$ is $2\pi/L$ for a given period $L\in\mathbb{R}_{>0}$, $i$ is the imaginary unit, $k_d$ is the $d$th item of $k$, and $\mathbf{q}_d$ the $d$th item of $\mathbf{q}$.
$\mathcal{T}$ is built so that the integration is between $\tau=t_0$ and $t$, and the notation $[D]$ indicates positive naturals up to $D$.

For the purposes of deriving the coefficients $\phi_k$, let us consider the GMM model in Eq.~(\ref{eq:gmm}) on a search space $\mathcal{Q}$. The space is further bounded to a symmetric set $[-L/2,L/2]^D$ since the Gaussians are symmetric about the zero axes. The resulting new model is then
\begin{equation}\small
  \Phi(\boldsymbol{\delta},\mathbf{q}(t)):=\sum_{d\in[2^D]}\sum_{k=1}^{n}\delta_k\,\mathcal{N}(\mathbf{q}(t)\,|\,A_d\mu_k,A_d\Sigma_k A_d^T)/2^D,
\end{equation}
where $A_d\in\mathbb{R}^{D\times D}$ are linear transformation matrices~\cite{calinon2020mixture}.


\subsection{Battery modeling}
\noindent

---

%%%%%%%%%%%%%%%%%%%%%%%%%%%%%%
\section{Experimental Results}
\noindent

---

\begin{figure*}[p]
  \begin{minipage}[t]{1\columnwidth}

    \input{figures/trajs.pdf_tex}
  \end{minipage}
  \hspace{.42cm}
  \begin{minipage}[t]{.93\columnwidth}
    \vspace*{-4.2cm}
    \caption{.   .   .   .   .   .   .   .   .   .   .   .   .   .   .   .   .   .   .   .   .   .   .   .   .   .   .   .   .   .   .   .   .   .   .   .   .   .   .   .   .   .   .   .   .   .   .   .   .   .   .   .   .   .   .   .   .   .   .   .   .   .   .   .   .   .   .   .   .   .   .   .   .   .   .   .   .   .   .   .   .   .   .   .   .   .   .   .   .   .   .   .   .   .   .   .   .   .   .   .   .   .   .   .   .   .   .   .   .   .   .   .   .   .   .   .   .   .   .   .   .   .   .   .   .   .   .   .   .   .   .   .   .   .   .   .   .   .   .   .   .   .   .   .   .   .   .   .   .   .   .   .   .   .   .   .   .   .   .   .   .   .   .   .   .   .   .   .   .   .   .   .   .   .   .   .   .   .   .   .   .   .   .   .   .   .   .   .   .   .   .   .   .   .   .   .   .   .   .   .   .   .   .   .   .   .   .   .   .   .   .   .   .   .   .   .   .   .   .   .   .   .   .   .   .   .   .   .   .   .   .   .   .   .   .   .   .   .   .   .   .   .   .   .   .   .   .   .   .   .   .   .   .   .   .   .   .   .   .   .   .   .   .   .   .   .   .   .   .   .   .   .   .   .   .   .   .   .   .   .   .   .   .   .   .   .   .   .   .   .   .   .   .   .   .   .   .   .   .   .   .   .   .   .   .   .   .   .   .   .   .   .   .   .   .   .   .   .   .   .   .   .   .   .   .   .   .   .   .   .   .   .   .   .   .   .   .   .   .   .   .   .   .   .   .   .   .   .   .   .   .   .   .   .   .   .   .   .   .   .   .   .   .   .   .   .   .   .   .   .   .   .   .   .   .   .   .   .   .   .   .   .   .   .   .   .   .   .   .   .   .   .   .   .   .   .   .   .   .   .   .   .   .   .   .   .   .   .   .   .   .   .   .   .   .   .   .   .   .   .   .   .   .   .   .   .   .   .   .   .   .   .   .   .   .   .   .   .   .   .   .   .   .   .   .   .   .   .   .   .   .   .   .   .   .   .   .   .   .   .   .   .   .   .   .   .   .   .   .   .   .   .   .   .   .   .   .   .   .   .   .   .   .   .   .   .   .   .   .   .   .   .   .   .   .   .   .   .   .   .   .   .   .   .   .   .   .   .   .   .   .   .   .   .   .   .   .   .   .   .   .   .   .   .   .   .   .   .   .   .   .   .   .   .   .   .   .   .   .   .   .   .   .   .   .   .   .   .   .   .   .   .   .   .   .   .   .   .   .   .   .   .   .   .   .   .   .   .   .   .   .   .   .   .   .   .   .   .   .   .   .   .   .   .   .   .   .   .   .   .   .   .   .   .   .   .   .   .   .   .   .   .   .   .   .   .   .   .   .   .   .   .   .   .   .   .   .   .   .   .   .   .   .   .   .   .   .   .   .   .   .   .   .   .   .   .   .   .   .   .   .   .   .   .   .   .   .   .   .   .   .   .   .   .   .   .}
  \end{minipage}
\end{figure*}

%%%%%%%%%%%%%%%%%%%%%%%%%%%%%%%%%%%%%%%%%%
\section{Conclusion and Future Directions}
\noindent

---


{\small
\bibliographystyle{IEEEtran} 
\bibliography{enerergo}
}

\end{document}


